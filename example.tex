\documentclass[%
  a4paper,
%  letterpaper,
%  onecolumn,
  twocolumn,
%  hidelinks,
%  linenumbers,
  algotwoe,
%  algopc
]{preprint}

%%%%%%%%%%%%%%%%%%%%%%%%%%%%%%%%%%%%%%%%%%%%%%%%%%%%%%%%%%%%%%%%%%%%%%%%%%%%%%%%
% INDIVIDUAL PACKAGES.                                                         %
%%%%%%%%%%%%%%%%%%%%%%%%%%%%%%%%%%%%%%%%%%%%%%%%%%%%%%%%%%%%%%%%%%%%%%%%%%%%%%%%

% Proper hyphenation.
\usepackage[english]{babel}

% Graphical packages.
\usepackage{graphicx}

% Math packages.
\usepackage{amsmath}
\usepackage{amssymb}
\usepackage{amsthm}


%%%%%%%%%%%%%%%%%%%%%%%%%%%%%%%%%%%%%%%%%%%%%%%%%%%%%%%%%%%%%%%%%%%%%%%%%%%%%%%%
% MAIN DOCUMENT.                                                               %
%%%%%%%%%%%%%%%%%%%%%%%%%%%%%%%%%%%%%%%%%%%%%%%%%%%%%%%%%%%%%%%%%%%%%%%%%%%%%%%%


\begin{document}
  
%%%%%%%%%%%%%%%%%%%%%%%%%%%%%%%%%%%%%%%%%%%%%%%%%%%%%%%%%%%%%%%%%%%%%%%%%%%%%%%%
% PAPER INFORMATION.                                                           %
%%%%%%%%%%%%%%%%%%%%%%%%%%%%%%%%%%%%%%%%%%%%%%%%%%%%%%%%%%%%%%%%%%%%%%%%%%%%%%%%

\title{Preprint Template Title}
  
\author[$\ast$]{First Author}
\affil[$\ast$]{Address of first author.\authorcr
  \email{f.author@email.com}, \orcid{0000-0000-0000-0000}}
  
\author[$\dagger$]{Second Author}
\affil[$\dagger$]{Address of second author.\authorcr
  \email{s.author@email.com}, \orcid{0000-0000-0000-0000}}
  
\shorttitle{Example short title}
\shortauthor{F. Author, S. Author}
\shortdate{}
\shortinstitute{}
  
\keywords{keyword1, keyword2, keyword3}

\msc{MSC1, MSC2, MSC3}
  
\abstract{%
  This preprint template was created by Steffen W. R. Werner for consistent
  writing of research papers.
  It yields a suitable default for preprints uploaded to open servers like
  arXiv and may be used for that purpose.
  In the following, the use of the template is explained in more details
  with some example for easy use.
}

\novelty{Here goes the actual novelty!}

\maketitle


%%%%%%%%%%%%%%%%%%%%%%%%%%%%%%%%%%%%%%%%%%%%%%%%%%%%%%%%%%%%%%%%%%%%%%%%%%%%%%%%
% PAPER CONTENT.                                                               %
%%%%%%%%%%%%%%%%%%%%%%%%%%%%%%%%%%%%%%%%%%%%%%%%%%%%%%%%%%%%%%%%%%%%%%%%%%%%%%%%
  
\section{Introduction}%
\label{sec:intro}

In the following sections, I give guidelines on the preprint commands
and options as well as a few examples how to use certain things.
This should not be seen as a general introduction on
``How to write papers with \LaTeX''.
Also, the TeX file corresponding to this document can be used as starting point
for writing your own paper in the preprint style.

In \Cref{sec:pkgcmd}, the most important points, hidden in the class file, are
described, where \Cref{subsec:options} contains main options that can be used
for the template, \Cref{subsec:linkpkg} the link packages used by the 
class file, \Cref{subsec:linenumbers} infos on optional line numbering and
\Cref{subsec:meta} the (default) commands for meta information.
\Cref{sec:export} is a small summary of the necessary files for using the
preprint template on other machines and \Cref{sec:other} is then used for
completeness of this file as a minimal example for preprints in this style.


%%%%%%%%%%%%%%%%%%%%%%%%%%%%%%%%%%%%%%%%%%%%%%%%%%%%%%%%%%%%%%%%%%%%%%%%%%%%%%%%
%%%%%%%%%%%%%%%%%%%%%%%%%%%%%%%%%%%%%%%%%%%%%%%%%%%%%%%%%%%%%%%%%%%%%%%%%%%%%%%%

\section{Packages and commands}%
\label{sec:pkgcmd}

Let's talk here about the options, packages and commands in the template.

%%%%%%%%%%%%%%%%%%%%%%%%%%%%%%%%%%%%%%%%%%%%%%%%%%%%%%%%%%%%%%%%%%%%%%%%%%%%%%%%

\subsection{Class options}%
\label{subsec:options}

The following is an overview about supported options for the template that can
be given to the document class to change its behavior:

\begin{description}
  \item[a4paper] Activates the default European A4 paper size.
  \item[letterpaper] Activates the default American US letter paper size.
  \item[onecolumn] Activates the single column style. \texttt{onecolumn}
    as option is equivalent to not using \texttt{twocolumn}.
  \item[twocolumn] Activates the double column style. If not set the 
    \texttt{onecolumn} style is the default.
  \item[hidelinks] Turns off the coloring of any link in the paper, i.e., all
    links are printed in default text color.
  \item[linenumbers] Activates line numbering in text.
  \item[algotwoe] Loads the \texttt{algorithm2e} package with preset style
    options for adding algorithm environments.
  \item[algops] Loads the \texttt{algorithm} and \texttt{algpseudocode} packages
    for adding algorithm environments.
\end{description}

As example, to use the preprint in double column style on A4 paper and with 
colored links you have to call:
\begin{verbatim}
  \documentclass[a4paper,twocolumn]{preprint}
\end{verbatim}
at the beginning of the main TeX file.

%%%%%%%%%%%%%%%%%%%%%%%%%%%%%%%%%%%%%%%%%%%%%%%%%%%%%%%%%%%%%%%%%%%%%%%%%%%%%%%%

\subsection{Links and hyperref packages}%
\label{subsec:linkpkg}

For links of internal and external kind, the template loads the packages
\texttt{hyperref}, \texttt{url}, \texttt{doi} and \texttt{cleveref}.
This allows the use of classical link commands as \textbackslash\texttt{href},
\textbackslash\texttt{url} and \textbackslash\texttt{doi} if needed and
their use by bibliography styles.
For distinction, the external links are colored in green, the internal citation
links in red and any internal reference will be given in blue.
Since the coloring of links would also appear in print versions, there is the
option for turning off the coloring; see \Cref{subsec:options}.

For internal referencing, the commands from \texttt{cleveref} are recommended.
The package is preloaded with the \texttt{nameinlink} and \texttt{capitalize} 
options such that names of environments are also part of the link and start with
uppercase letters.
Use \verb|\Cref{...}| for any named environment as sections, figures, 
algorithms, tables or equations at the beginning of sentences and 
\verb|\cref{...}| for default equation referencing in the text.
By this, the environments are automatically put in by their right naming and
number, e.g., if we want to reference to two sections we use
\begin{verbatim}
  \Cref{sec1,sec2}
\end{verbatim}
which gives could then give
\begin{center}
  \Cref{sec:intro,sec:pkgcmd}
\end{center}
automatically, or for example with the range of three sections
\begin{verbatim}
  \Cref{sec1,sec2,sec3}
\end{verbatim}
to get
\begin{center}
  \Cref{sec:intro,sec:pkgcmd,sec:export}.
\end{center}

%%%%%%%%%%%%%%%%%%%%%%%%%%%%%%%%%%%%%%%%%%%%%%%%%%%%%%%%%%%%%%%%%%%%%%%%%%%%%%%%

\subsection{Line numbering}%
\label{subsec:linenumbers}

The line numbering of the text, which is turned on by the class option
\textbf{\textsf{linenumbers}}, uses the \verb|lineno| package.
This has by default problems with math environments especially from other
packages like \verb|amsmath|.
Therefore, a patch is integrated to automatically fix the line numbering for all
math environments from plain \LaTeX and the \verb|amsmath| package.
If you want to use the line numbering with other math environments, you can
automatically patch those using the
\begin{verbatim}
  \patchMATHlinenumbers{...}
\end{verbatim}
command for allowing line numbering for your environment, e.g.,
\begin{verbatim}
  \patchMATHlinenumbers{align}
\end{verbatim}
is used in the class file to patch the \verb|align| environment from
\verb|amsmath|.
For other customization of the line numbering, have a look at the package
documentation of the \verb|lineno| package.

%%%%%%%%%%%%%%%%%%%%%%%%%%%%%%%%%%%%%%%%%%%%%%%%%%%%%%%%%%%%%%%%%%%%%%%%%%%%%%%%

\subsection{Commands for basic paper information}%
\label{subsec:meta}

The template allows you to use a bunch of (default) article commands that should
be used according to this example paper.

\begin{description}
  \item[\normalfont\textbackslash\texttt{title}\{\ldots\}] Defines the title
    of the article.
  \item[\normalfont\textbackslash\texttt{author}{[\ldots]}\{\ldots\}] Defines
    an author of the paper. The optional argument allows to set a footnote mark
    as number or symbol (e.g., 1 or \verb|$\ast$|) to associate authors with
    affiliations. These authors should be given with full first and last names.
    The order of the authors in the TeX document determines the order of the
    authors on the paper.
  \item[\normalfont\textbackslash\texttt{affil}{[\ldots]}\{\ldots\}] Defines
    affiliation for the authors. The optional argument allows to set a footnote
    mark as number or symbol (e.g., 1 or \verb|$\ast$|) to associate authors
    with affiliations. The affiliation should have the following format
\begin{verbatim}
  Address.\authorcr
  \email{...}, \orcid{...}
\end{verbatim}
    such that email and ORCID are automatically formatted and linked.
  \item[\normalfont\textbackslash\texttt{shorttitle}\{\ldots\}] Is the optional
    running title of the paper printed on all pages starting with 2.
  \item[\normalfont\textbackslash\texttt{shortauthor}\{\ldots\}] Is the optional
    running authors of the paper printed on all pages starting with 2.
    Here, abbreviated first names should be used. For more then
    3 authors, it's recommended to use the first author abbreviated and followed
    by an ``et al.'' (without quotation marks).
  \item[\normalfont\textbackslash\texttt{shortdate}\{\ldots\}] Controls the
    content of the dates field in the lower right corner. If empty, the
    compilation date is used in ISO format (yyyy-mm-dd). Hiding the date
    is also possible by, e.g., using a non-breakable space 
    \verb|\shortdate{|$\sim$\verb|}|.
  \item[\normalfont\textbackslash\texttt{shortinstitute}\{\ldots\}] Allows
    to set an optional institution in brackets behind ``Preprint'' in the lower
    left corner.
  \item[\normalfont\textbackslash\texttt{keywords}\{\ldots\}] Comma separated
    list of keywords. If not set, the bold face \textbf{Keywords:} below the 
    abstract will not be shown.
  \item[\normalfont\textbackslash\texttt{msc}\{\ldots\}] Comma separated
    list of math subject classification identifiers. If not set, the bold face
    \textbf{Mathematics subject classification:} below the abstract will not be
    shown.
    See for example
    \begin{center}
      \url{https://zbmath.org/classification/}
    \end{center}
    for the list of MSC identifiers.
  \item[\normalfont\textbackslash\texttt{abstract}\{\ldots\}] Shows the abstract 
    text. 
    Note that this is here only a command and not the classical abstract
    environment.
\end{description}


%%%%%%%%%%%%%%%%%%%%%%%%%%%%%%%%%%%%%%%%%%%%%%%%%%%%%%%%%%%%%%%%%%%%%%%%%%%%%%%%
%%%%%%%%%%%%%%%%%%%%%%%%%%%%%%%%%%%%%%%%%%%%%%%%%%%%%%%%%%%%%%%%%%%%%%%%%%%%%%%%

\section{Export template for preprint server}%
\label{sec:export}

To use the preprint on other machines, e.g., to give it to your co-authors or
for the upload to a preprint server like \emph{arXiv}, the
\texttt{mypreprint.cls} file needs to be copied wherever needed.
No further files are required.
To compile the template, only basic \LaTeX packages given in any
minimal \emph{TeX Live} or \emph{MiKTeX} installation are needed.


%%%%%%%%%%%%%%%%%%%%%%%%%%%%%%%%%%%%%%%%%%%%%%%%%%%%%%%%%%%%%%%%%%%%%%%%%%%%%%%%
%%%%%%%%%%%%%%%%%%%%%%%%%%%%%%%%%%%%%%%%%%%%%%%%%%%%%%%%%%%%%%%%%%%%%%%%%%%%%%%%

\section{Other \LaTeX{} stuff}%
\label{sec:other}

Just for the completeness of the template example, here come a few of the usual
things you see in other journal styles.
Basically all following points are just reminders of basic \LaTeX for paper
writing that also works in the preprint class.

%%%%%%%%%%%%%%%%%%%%%%%%%%%%%%%%%%%%%%%%%%%%%%%%%%%%%%%%%%%%%%%%%%%%%%%%%%%%%%%%

\subsection{Math environments}%
\label{subsec:math}

For proper referencing of equations, the template loads by default the
\verb|amsmath| package.
The rest can be loaded as desired.
Here, we will just demonstrate to reference equations with the \verb|cleveref|
package.
Given those three equations
\begin{align}
  \label{eqn:eqn1}
  x = a + b,\\
  \label{eqn:eqn2}
  c = \frac{1}{2 x},\\
  \label{eqn:eqn3}
  V = \int\limits_{0}^{\infty} CB \mathrm{d}t.
\end{align}
Then \verb|\cref| should be used instead of \verb|\eqref| for referring to
a single equation~\cref{eqn:eqn1} by
\begin{verbatim}
  \cref{eqn:eqn1}
\end{verbatim}
to a set of equations~\cref{eqn:eqn1,eqn:eqn3}
\begin{verbatim}
  \cref{eqn:eqn1,eqn:eqn3}
\end{verbatim}
and to a range of equations~\cref{eqn:eqn1,eqn:eqn2,eqn:eqn3} by
\begin{verbatim}
  \cref{eqn:eqn1,eqn,eqn:2,eqn:eqn3}.
\end{verbatim}
Do not forget to put an unbreakable space \~{} between the naming and
\verb|\cref|.

In case you actually need the environment written with the reference number
use the uppercase \verb|\Cref| as for other environments, e.g.,
\Cref{eqn:eqn1,eqn:eqn2,eqn:eqn3} is generated by
\begin{verbatim}
  \Cref{eqn:eqn1,eqn,eqn:2,eqn:eqn3}.
\end{verbatim}

%%%%%%%%%%%%%%%%%%%%%%%%%%%%%%%%%%%%%%%%%%%%%%%%%%%%%%%%%%%%%%%%%%%%%%%%%%%%%%%%

\subsection{Reference lists}%
\label{subsec:ref}

For the bibliography style it is recommended to use something that supports
DOIs.
In this example, the \verb|plainurl| style is used.
Citations can be done as usual using the \verb|\cite| command.
The template is preloading the \verb|cite| package with the option
\verb|noadjust|.
This allows automatic ordering of references in citation brackets and conversion
to ranges of references if possible.

Citation of a single reference is then~\cite{ref1} with
\begin{verbatim}
  \cite{ref1}
\end{verbatim}
for two references in the wrong order~\cite{ref3, ref1}
\begin{verbatim}
  \cite{ref3,ref1}
\end{verbatim}
and for ranges of references in any order~\cite{ref2, ref1, ref3}
\begin{verbatim}
  \cite{ref2,ref1,ref3}
\end{verbatim}

Remember to adjust the loaded BibTeX file to your needs and do not forget to
put an unbreakable space \~{} between the naming and \verb|\cite|.
Additionally, the references are added to the table of contents by
\begin{verbatim}
  \addcontentsline{toc}{section}{References}
\end{verbatim}
to give proper links in PDF viewers.

%%%%%%%%%%%%%%%%%%%%%%%%%%%%%%%%%%%%%%%%%%%%%%%%%%%%%%%%%%%%%%%%%%%%%%%%%%%%%%%%

\subsection{Graphics and TikZ}%
\label{subsec:graphics}

\begin{figure}[t]
  \centering
  \resizebox{8cm}{!}{
  \includegraphics{example-image-a}}
   
  \caption{Example figure for template.}
  \label{fig:example}
\end{figure}

No special mentioning of fancy graphics or use of TikZ here.
Just a simple example how an included graphic should look like with the 
corresponding \verb|cleveref| reference \Cref{fig:example}.


%%%%%%%%%%%%%%%%%%%%%%%%%%%%%%%%%%%%%%%%%%%%%%%%%%%%%%%%%%%%%%%%%%%%%%%%%%%%%%%%

\subsection{Tables}%
\label{subsec:tables}

The same as for figures now with an example table \Cref{tab:example}.
For demonstration, also an intermediate breaking horizontal line in the content 
is added.
Note that table captions should go above the actual table.

\begin{table}[t]
  \centering
  \caption{Example table}
  \label{tab:example}
  
  \begin{tabular}{lcr}
    \hline\noalign{\medskip}
    & header1 & header2\\
    \noalign{\smallskip}\hline\noalign{\medskip}
    row1 & c1 & c2 \\
    row2 & c3 & c4 \\
    \noalign{\smallskip}\hline\noalign{\medskip}
    row3 & c5 & c6 \\
    \noalign{\medskip}\hline\noalign{\smallskip}
  \end{tabular}
\end{table}

%%%%%%%%%%%%%%%%%%%%%%%%%%%%%%%%%%%%%%%%%%%%%%%%%%%%%%%%%%%%%%%%%%%%%%%%%%%%%%%%

\subsection{Algorithms}%
\label{subsec:algorithms}

This template has two integrated options for adding algorithm environments:
(i) \texttt{algotwoe} loads the \texttt{algorithm2e} package with style
options, and (ii) \texttt{algops} loads the \texttt{algorithm} and
\texttt{algpseudocode} packages.
The loading of these packages inside the template ensures that the counter
definitions for referencing with \texttt{cleveref} are correctly defined.

An example for an algorithm environment is given in \Cref{alg:test}.
By default, the variant using \texttt{algorithm2e} is shown, but an analogous
version for the other algorithm packages is commented out in the
\LaTeX{} code.
Due to the correct loading sequence it is possible to refer to the single
lines of the algorithm using labels.
For example in \Cref{alg:test}, we have \Cref{alg:line1} and
\Cref{alg:line2,alg:line3,alg:line4,alg:line5}.

\begin{algorithm}[t]
  \SetAlgoHangIndent{1pt}
  \DontPrintSemicolon
  \caption{Test algorithm.}
  \label{alg:test}
  
  \KwIn{Some quantities to give to the algorithm.}
  \KwOut{Computed things.}
  
  First computational step.\;
  \label{alg:line1}
  
  \For{cond \label{alg:line2}}{
    A second step that ends with a formula
      \begin{equation*}
        c = a + b.
      \end{equation*}\vspace{-\baselineskip}\;
      \label{alg:line3}
    
    And a last step where the formula
      \begin{equation*}
        x = \sqrt{2}
      \end{equation*}
      is in the middle of the sentence.\;
      \label{alg:line4}
  } \label{alg:line5}
\end{algorithm}

%\begin{algorithm}
%  \caption{Test algorithm.}
%  \label{alg:test}
%  
%  \hspace*{0\baselineskip}\textbf{Input:}~Some quantities to give to the
%    algorithm.
%    
%  \hspace*{0\baselineskip}\textbf{Output:}~Computed things.
%  
%  \begin{algorithmic}[1]
%    \State First computational step.
%      \label{alg:line1}
%      
%    \For{cond}
%      \State A second step that ends with a formula
%        \begin{equation*}
%          c = a + b.
%        \end{equation*}\vspace{-\baselineskip}\;
%        \label{alg:line2}
%        
%      \State And a last step where the formula
%        \begin{equation*}
%          x = \sqrt{2}
%        \end{equation*}
%        \hspace{\baselineskip}
%        is in the middle of the sentence.
%        \label{alg:line3}
%    \EndFor
%  \end{algorithmic}
%\end{algorithm}

%%%%%%%%%%%%%%%%%%%%%%%%%%%%%%%%%%%%%%%%%%%%%%%%%%%%%%%%%%%%%%%%%%%%%%%%%%%%%%%%

\subsection{Numerical experiment sections}%
\label{subsec:numexp}

When using numerical experiments in a paper, the reproducibility of the
results is extremely important.
Therefore, you need to mention all used hardware and software in the
introduction of the numerical examples section.
For even better scientific practice, the source codes or scripts used in the
computations should become available, e.g., by uploading on Zenodo.
In this case, a code availability block should be added to the paper.
An example for such a block can be seen below.

\begin{center}%
  \setlength{\fboxsep}{5pt}%
  \fbox{%
  \begin{minipage}{.92\linewidth}
    \textbf{Code availability}\newline
    The source codes and scripts used to compute the results presented in this 
    paper can be obtained from
    \begin{center}
      \href{https://doi.org/?????/????????}%
        {\texttt{doi:?????/????????}}
    \end{center}
    under the ??? license and authored by ???.
  \end{minipage}}
\end{center}

A less standing out variant is to incorporate the above block as sentence with
a reference to the code package.
This can easily be copied by readers as reference in their works.


%%%%%%%%%%%%%%%%%%%%%%%%%%%%%%%%%%%%%%%%%%%%%%%%%%%%%%%%%%%%%%%%%%%%%%%%%%%%%%%%
% *** ACKNOWLEDGEMENTS ***                                                     %
%%%%%%%%%%%%%%%%%%%%%%%%%%%%%%%%%%%%%%%%%%%%%%%%%%%%%%%%%%%%%%%%%%%%%%%%%%%%%%%%

\section*{Acknowledgments}%
\addcontentsline{toc}{section}{Acknowledgments}

Towards the end, there can be an unnumbered section for the acknowledgments,
stating people and organizations who influenced or funded the work in the paper.
Note that this section is just created via
\begin{verbatim}
\section*{Acknowledgments}%
\addcontentsline{toc}{section}{Acknowledgments}
\end{verbatim}


%%%%%%%%%%%%%%%%%%%%%%%%%%%%%%%%%%%%%%%%%%%%%%%%%%%%%%%%%%%%%%%%%%%%%%%%%%%%%%%%
% *** REFERENCES ***                                                           %
%%%%%%%%%%%%%%%%%%%%%%%%%%%%%%%%%%%%%%%%%%%%%%%%%%%%%%%%%%%%%%%%%%%%%%%%%%%%%%%%

\addcontentsline{toc}{section}{References}
\bibliographystyle{plainurl}
\bibliography{exampleref}
  
\end{document}
